%region メタ情報・前文
\documentclass[master]{cimt}
% オプションについては,マニュアルを参照.
% \documentclass[master,oneside]{cimt} 

% 必要とするパッケージがあれば,ここで指定する.
\usepackage[dvipdfmx]{graphicx,color}
\usepackage{url}
\usepackage{layout}
\usepackage[fleqn]{amsmath}
\usepackage[psamsfonts]{amssymb}
\usepackage{multirow}
\usepackage{listings,jlisting,upquote}
\usepackage{algorithm}
\usepackage{algorithmic}
\usepackage{color}

\renewcommand{\algorithmicrequire}{\textbf{Input:}}
\renewcommand{\algorithmicensure}{\textbf{Output:}}


\newcommand{\argmin}{\operatornamewithlimits{argmin}}
\newcommand{\argmax}{\operatornamewithlimits{argmax}}

\newcommand{\todo}[1]{{\color{red} ToDo: {#1}}}
\newcommand{\com}[1]{{\color{blue}(#1)}}

\usepackage[dvipdfmx,bookmarks=true,setpagesize=false]{hyperref}
\usepackage{pxjahyper} % http://konoyonohana.blog.fc2.com/blog-entry-68.html
\hypersetup{
  bookmarksopen=true,
  bookmarksnumbered=true,
  bookmarksopenlevel=1,
  colorlinks=true,
  linkcolor=black,
  citecolor=black,
  filecolor=black,
  menucolor=magenta,
  urlcolor=blue,
  pdftitle={},
  pdfauthor={ISI}
}

% 論文タイトル
\jtitle{病理画像解析支援のための深層能動学習に関する研究}
\etitle{Cost effective learning of pathological images with deep active learning}
\jauthor{福田 圭佑}
\eauthor{Keisuke Fukuta}
\supervisor{原田 達也 教授}
\handin{2018}{1} % 提出月 2018/1

\begin{document}

\maketitle 

\frontmatter{}

\begin{jabstract}
  \input jabst.tex
\end{jabstract}

\begin{eabstract}
  \input eabst.tex
\end{eabstract}

\setcounter{tocdepth}{2}
\tableofcontents

\mainmatter{}

\input 1_intro.tex
\input 2_related.tex
\input 3_al.tex
\input 4_method.tex
\input 5_experiment.tex
\input 6_concl.tex

%region 後付
\backmatter{}

% 参考文献: BibTeX を使う場合の例 (styleは適宜選択)
\bibliographystyle{junsrt}
\bibliography{mendeley, reference}

% 謝辞 (前文においても良い)
\begin{acknowledgements}
TODO
\end{acknowledgements}

%付録 (必要な場合のみ)
\appendix

\chapter{ソースコード}
  \begin{verbatim}
    int main () {
      ...
      ...
    }
  \end{verbatim}

\end{document}
%endregion
%endregion
