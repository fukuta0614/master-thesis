\chapter{序論}

\section{研究の背景}
\com{デジタルパソロジー 診断補助の可能性}

病理標本のデジタル画像を撮影し,ディスプレイに表示することで細胞の組織を観察して診断を行うデジタルパソロジー (Digital Pathology)の普及が進んでいる.
病理画像は1スライド当たりギガピクセルに及ぶ非常に高解像画像で,組織すべてを隅々まで確認して異常がないか診断を行うのは熟練の病理医でも時間を要する.
一方,デジタルパソロジーの普及に伴い組織の画像データが大量に保管されるようになり,画像解析技術の応用によって病理医の診断を補助するための研究が数多く行われている.
\cite{gurcan2009histopathological},\todo{引用}

\com{機械学習による診断, 特にCNN, これらはめちゃくちゃデータ食う}

\cite{doyle2008automated}, \cite{dundar2011computerized}

機械学習手法を用い,汎化性能の高い識別器を教師付き学習によって構築するには大量のラベル付きデータが必要となることが多い.
また,近年の画像認識分野において,Convolutional Neural Network (CNN)と呼ばれる多層ニューラルネットワークを用いた機械学習技術による成果が目覚ましい.
従来の画像認識では人手によって特徴量を設計し,その特徴量を用いて識別器を学習していたのに対し,
CNNは学習の過程で訓練データから識別に有効な特徴量を抽出する表現学習と識別器の学習を同時に行うことができる点が特徴的である.

病理画像認識の分野においてもCNNを使用した研究は急速に増加しており,組織の異常や病変の認識,検出の精度は熟練の病理医に匹敵することが示されている.
1スライド当たりギガピクセルに及ぶ病理画像解析では,スライドを複数の画像パッチに分割し,それらに対して識別及びセグメンテーションを行う場合が多い.
我々は,Camelyon17という乳癌のリンパ節転移を検出するタスクにおいて,CNNの共分散特徴量を利用した学習器によって,病理医に近い精度を達成した.
また,機械による識別結果を利用しながら病理医が診断を行った場合,機械のみ,病理医のみによる診断と比較して速度,精度ともに向上したという報告がある.

\com{医療画像のようにアノテーションコスト高いと厳しい(ここでcamelyon)} %%%
これらのことから,CNNを病理画像に利用するのは非常に有望であると考えられるが,これらの学習には大量のラベル,またはアノテーションを必要であるという大きな欠点がある.
当然のことながら病理画像の学習データの作成には高度に専門的な知識を必要とするため,ギガピクセルに及ぶ画像の詳細なアノテーションの付与には莫大なコストが伴う.


\com{アノテーションコストが高い問題を緩和するための手法の一つにactive learningがある} %%%
ラベルが不足した状況に対する機械学習のアプローチとして,半教師付き学習や教師なし学習などが挙げられるが,教師あり学習と比較して精度の点で大きく劣る場合が多い.

教師付き学習は,入力(質問)と出力(答え)の組からなる訓練データを用いて, その背後に潜んでいる入出力関係(関数)を学習する問題である
教師なし学習は,文字通り教師がいない状況での学習であり, 出力(答え)の無い入力データのみが与えられる. 教師なし学習の目的は状況によって異なり,数学的にきちんと定式化できない場合が多い. 例えば,入力データの似たもの同士をグループ化するクラスタリングがその典型的な例である
教師付き学習では入力と出力の組からなる訓練データが与えられ, 教師なし学習では入力だけの訓練データが与えられる. 
半教師付き学習は,これらの中間の状況に対応する学習問題であり, 入力と出力の組からなる訓練データに加え入力だけの訓練データも与えられる. 
半教師付き学習の目標は,教師付き学習と同じく高い汎化能力を獲得することである. 
半教師付き学習では, 入出力両方が揃っている訓練データの数は少なく, 入力だけの訓練データの数は非常に多い場合を考えるのが典型的である. 
このような状況では,少数の入出力データだけでなく多数の入力データも用いる事により, より高い汎化能力が獲得できると期待される.

能動学習は

そこで本研究では,病理医が支払うコストを最小にしつつ精度を担保するためのアプローチとして,
学習データのラベルを学習の過程で機械とアノテーターとのinteractionを通じて増加させていくActive Learningを病理画像解析に取り入れることを提案する.

\com{activeの説明 理論的にもいろいろあるぜ}

\com{active + Pathology}
sample, i.i.dじゃないよね.似た領域には似たサンプルがあるよね
enforce diversity 

\com{active + deep}

このような学習方法が浸透するにつれ,大規模データベースに対するアノテーションコストは増加している.


\section{研究の目的}

\section{研究の構成}

\section{研究の貢献}



