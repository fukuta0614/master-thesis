\chapter{序論}

\section{本研究の背景}

\com{デジタルパソロジー 診断補助の可能性}
病理診断は、病理医がスライドガラス上の標本を観察することにより行われる。
近年では標本全体のデジタル画像を撮影し,ディスプレイに表示することで細胞の組織を観察して診断を行うデジタルパソロジーの普及が進んでいる\cite{pantanowitz2010digital}.
このデジタル画像はWhole slide image, WSIと呼ばれる。
WSIは1スライド当たりギガピクセルに及ぶ非常に高解像画像で,組織すべてを隅々まで確認して異常がないか診断を行うのは熟練の病理医でも時間を要する.

\com{機械学習による診断}
一方,デジタルパソロジーの普及に伴ってWSIが大量に保管されるようになり,機械学習を利用した画像解析技術の応用によって病理医の診断を補助するための研究が数多く行われている\cite{gurcan2009histopathological},\cite{komuraishikawa}、\cite{litjens2017survey}。
中でも、WSIを入力として対応する病名を出力する自動診断に関する研究が盛んに行われている\cite{doyle2008automated}、 \cite{dundar2011computerized}。

\com{特にCNN, これらはめちゃくちゃデータ食う}
しかし、近年の画像認識分野において目覚ましい成果を挙げているConvolutional Neural Network (CNN)をはじめとするように、機械学習手法を用い,
汎化性能の高い識別器を構築するには大量のラベル付きデータが必要となることが多い.

\com{医療画像のようにアノテーションコスト高いと厳しい(ここでcamelyon)} %%%
病理画像などの医療画像の学習データセットの作成には高度に専門的な知識が不可欠であるため,医師への大きな負担となり、医師にとって大きな負担となってしまいアノテーションコストが非常に高い.
また、特に病理画像においては、1枚ギガピクセルに及ぶWSIに対する詳細なアノテーションの付与には莫大なコストが伴う.
Camelyon Grand Challenge\cite{Camelyon17}にて公開されたCamelyon Datasetでは、1000枚にも及ぶWSIに詳細な乳癌のリンパ節転移が生じている領域にアノテーションが付与されており、
WSI1枚当たり熟練の病理医が1時間を費やしたという。

\com{アノテーションコストが高い問題を緩和するための手法の一つにactive learningがある} %%%
教師データ作成コストが問題となるのは医療画像解析に限った話ではなく、機械学習では少量のデータから効率的に学習する枠組みの研究も盛んに行われている。
ラベルが不足した状況に対する機械学習のアプローチとして,能動学習と呼ばれる枠組みがある。

教師付き学習は入力(質問)と出力(答え)の組からなる訓練データを用いて, その背後に潜んでいる入出力関係(関数)を学習する問題であり、しばしば問題集と解答集が与えられてテスト問題でできるだけ高い点を取る状況と例えられる。
また、半教師付き学習は問題集と一部の解答のみが与えられる状況だと言える。
ここで、能動学習とは、教師なし学習と同様に問題集のみが与えられるが、それとは別に教師に自分のわからないところを聞くことができる枠組みであると例えることが出来る。
能動学習は,「もし学習アルゴリズムが学習データ全体の中から任意のデータを選択することができる場合,適切な選択によって得られる学習器の性能は向上する」という仮説に基づいている。
すなわち、大量に与えられたラベルなしデータから中から、モデルの識別精度向上に寄与する可能性の高いサンプルを選択し、
オラクル(アノテーター、もしくはドメインの専門家)のみにアノテーションをその都度付与してもらうことで、
モデルの識別率を犠牲にすることなくアノテーションコストを最小化するためのアプローチである。
サンプル選択における理論的な研究だけでなく、実用的なアプリケーションに関する研究も近年増加しており、
高いアノテーションコストが大きな課題である病理画像解析への適用親和性は非常に高いと言える。


\section{本研究の目的}
本研究では,病理画像解析において病理医が支払うコストを最小にしつつモデルの識別精度を担保するためのアプローチとして,
学習データのラベルを学習の過程で機械とアノテーターとのinteractionを通じて増加させていく能動学習を採用したシステムの構築を目的とする。
能動学習を病理画像に適用するための工夫だけでなく,近年の画像認識分野で目覚ましい成果を挙げているConvolutional Neural Network (CNN)を
能動学習に適用するための新たな手法を提案し,その有効性を検証する。


\section{本研究の構成}
本論文の構成を以下に示す.\\
第1章で本研究の背景と目的を述べた。 \\
第2章では,本研究と関わりのある病理画像解析、深層能動学習の先行研究について述べる。\\
第3章では,能動学習の基本的な概念を述べる。\\
第4章では、本研究で提案する手法について述べる。\\
第5章では,実験の目的,設定,結果について述べる.\\
第6章では,本研究の結論,および展望について述べる.

\section{本研究の貢献}
\begin{itemize}
    \item 病理画像解析に能動学習を利用するための実用的な手法の提案
    \item 深層能動学習のための有効なサンプル選択の戦略の提案
\end{itemize}