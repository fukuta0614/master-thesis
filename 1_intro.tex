\chapter{序論}

\section{本研究の背景}

病理診断は、病理医がスライドガラス上の標本を観察することにより行われる。
近年では標本全体のデジタル画像を撮影し,ディスプレイに表示することで細胞の組織を観察して診断を行うデジタルパソロジーの普及が進んでいる\cite{pantanowitz2010digital}.
このデジタル画像はWhole slide image, WSIと呼ばれる。
WSIは1スライド当たりギガピクセルに及ぶ巨大な高解像画像で,組織すべてを隅々まで確認して異常がないか診断を行うのは熟練の病理医でも時間を要する.

一方,デジタルパソロジーの普及に伴ってWSIが大量に保管されるようになり,機械学習を利用した画像解析技術の応用によって病理医の診断を補助するための研究が数多く行われている\cite{gurcan2009histopathological, komuraishikawa, litjens2017survey}。
中でも、WSIを入力として対応する病名を出力する自動診断に関する研究が盛んに行われている\cite{doyle2008automated,dundar2011computerized}。

しかし、近年の画像認識分野において目覚ましい成果を挙げているConvolutional Neural Network (CNN)をはじめとするように、機械学習手法を用い,
汎化性能の高い識別器を構築するには大量のラベル付きデータが必要となることが多い.病理画像などの医療画像の学習データセットの作成には高度に専門的な知識が不可欠であるため,医師への大きな負担となり、アノテーションコストが非常に高い.
また、特に病理画像においては、1枚ギガピクセルに及ぶWSIに対する詳細なアノテーションの付与には莫大なコストが伴う.
Camelyon Datasetとは、Camelyon Grand Challenge\cite{Camelyon17}にて公開されたデータセットで、1000枚にも及ぶWSIに対して乳癌のリンパ節転移の存在する領域への詳細なアノテーションが付与されている。
WSI1枚のアノテーションにつき熟練の病理医が1時間かけて行ったとされ、延べ1000時間をこのデータセット作成に費やしたという報告もある。

教師データ作成コストが問題となるのは医療画像解析に限った話ではなく、機械学習では少量のデータから効率的に学習する枠組みの研究も盛んに行われている。
アノテーションコストが特に高い状況に対する機械学習のアプローチとして,能動学習と呼ばれる枠組みがある。

教師付き学習は入力(質問)と出力(答え)の組からなる訓練データを用いて, その背後に潜んでいる入出力関係(関数)を学習する問題であり、
しばしば問題集と解答集が与えられてテスト問題でできるだけ高い点数を取る状況に例えられる。
またその例えでは、半教師付き学習は問題集と一部の解答のみが与えられる状況だと言える。
ここで、能動学習とは、教師なし学習と同様に問題集のみが与えられるが、学習の過程でわからないところを教師に聞くことができる枠組みであると例えることが出来る。
つまり、大量に与えられたラベルなしデータの中から、モデルの識別精度向上に寄与する可能性の高いサンプル(クエリ)を選択し、
オラクル(アノテーター、もしくはドメインの専門家)にアノテーションをその都度付与してもらうことで、
アノテーションコストを最小化しつつモデルの識別率を向上させるアプローチである。
クエリの選考基準における理論的な研究だけでなく、実用的なアプリケーションに関する研究も近年増加しており、\todo{引用}
高いアノテーションコストが大きな課題である病理画像解析への適用親和性は非常に高いと言える。

しかしそれらの研究は、一部の例外を除き高次元のデータへのscalabilityに欠く物が多く、
計算コストの面からCNNのような巨大なパラメータを持つモデルを利用できない場合が多い。
また、能動学習の多くでは特徴量が変化しないという状況での問題設定を置いていることが多く、表現学習を同時に行う深層学習を用いた研究は少ない。
\todo{ここにもう一言書いてもいいかも}

\section{本研究の目的}
本研究では,病理画像解析において病理医が支払うコストを最小にしつつモデルの識別精度を担保するためのアプローチとして、
ラベル付き学習データを学習の過程で機械とアノテーターとのinteractionを通じて増加させていく能動学習を採用したシステムの構築を目的とする。
能動学習を病理画像に適用するための工夫のみならず、近年の画像認識分野で目覚ましい成果を挙げているConvolutional Neural Network (CNN)を
能動学習に適用するための新たな手法を提案し,その有効性を検証する。
\todo{ここに手法の中身書くもの?}


\section{本研究の構成}
本論文の構成を以下に示す.\\
第1章で本研究の背景と目的を述べた。 \\
第2章では,関連研究と本研究の位置付けについて述べる。\\
第3章では、本研究で提案する手法について述べる。\\
第4、5章では,提案した手法の有効性を確認するために行った実験の目的,設定,結果について述べる.\\
第6章では,本研究の結論,および展望について述べる.

\section{本研究の貢献}
\todo{ここもあやしい}
\begin{itemize}
    \item 病理画像解析に能動学習を利用するためのシステムの構築
    \item 深層能動学習のために有効でかつ実用的なクエリ選択の戦略 Query-By-Dropout-Predictionsを提案。
\end{itemize}
