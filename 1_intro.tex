\chapter{序論}

\section{本研究の背景}

病理診断は,病理医がスライドガラス上の標本を観察することにより行われる.
近年では標本全体のデジタル画像を撮影し,ディスプレイに表示することで細胞の組織を観察して診断を行うデジタルパソロジーの普及が進んでいる\cite{pantanowitz2010digital}.
このデジタル画像はWhole Slide Image(WSI)と呼ばれる.
WSIは1スライド当たりの画素数がギガピクセルに及ぶ巨大な高解像画像で,組織すべてを隅々まで確認して異常がないか診断を行うのは熟練の病理医でも時間を要する.

一方,デジタルパソロジーの普及に伴ってWSIが大量に保管されるようになり,機械学習を利用した画像解析技術の応用によって病理医の診断を補助するための研究が数多く行われている\cite{gurcan2009histopathological, komuraishikawa, litjens2017survey}.
中でも,WSIを入力として対応する病名を出力する自動診断に関する研究が盛んに行われている\cite{doyle2008automated,dundar2011computerized}.

しかし,機械学習手法を用いて汎化性能の高い識別器を構築するには大量のラベル付きデータが必要となることが多い.
これは近年の画像認識分野において目覚ましい成果を挙げているDeep Convolutional Neural Network(DCNN)などの深層学習では特に顕著である.

病理画像などの医療画像に対するラベルの付与には高度な専門的知識を要するため,全てのアノテーション作業を医師が行わなければならず,学習データセットの作成コストが非常に高い.
また,特に病理画像はWSI1枚あたりの画素数がギガピクセルに及ぶほど巨大であるため,詳細なアノテーションの付与には長時間に及ぶ集中力が求められる.
例えば,Camelyon Grand Challenge \cite{Camelyon17}にて公開されたCamelyon Datasetでは,1,000枚にも及ぶWSIに対して乳癌のリンパ節転移の存在する領域への詳細なアノテーションが付与されている.
これらのアノテーションはWSI1枚につき熟練の病理医が1時間かけて行ったとされ,延べ1,000時間をこのデータセット作成に費やしたという報告がある.

学習データセットの作成コストが問題となるのは医療画像解析に限らず,工場での異常検知や,詳細画像認識など様々な状況で起こる.
機械学習では、少量のデータから効率的に学習するための枠組みが存在する。
アノテーションコストが特に高い状況に対する機械学習のアプローチとして,能動学習\cite{settles2010active}と呼ばれる枠組みがある.

教師付き学習は入力(質問)と出力(答え)の組からなる訓練データを用いて, その背後に潜んでいる入出力関係(関数)を学習する問題であり,
しばしば問題集と解答集が与えられてテスト問題でできるだけ高い点数を取る状況に例えられる.
またこの例えを用いると,教師なし学習は問題集のみが与えられ解答が一切与えられない状況だと言える.
ここで,能動学習とは,教師なし学習と同様に問題集のみが与えられ,学習の過程で教わりたい問題の答えだけを教師に聞きながら学習を行う枠組みであると例えることが出来る.
つまり,大量に与えられたラベルなしデータの中から,モデルの識別精度向上に寄与する可能性の高いサンプルを選択し,
オラクル(アノテーター,もしくはドメインの専門家)にアノテーションを付与してもらい,そのデータを利用して再学習を行うというサイクルを繰り返すことで
アノテーションコストを抑えつつモデルの識別率を向上させるアプローチである.
クエリの選考基準における理論的な研究だけでなく,実用的なアプリケーションに関する研究も年々増加しており,
高いアノテーションコストが大きな課題である病理画像解析への適用親和性は非常に高いと言える.

しかし能動学習の研究は,一部の例外を除き高次元のデータへのスケーラビリティに欠く手法が多い.
病理画像では,WSI1枚あたりに大量の画像パッチを含むうえに,それらの画像パッチ1枚1枚も高次元な画像データであるため,扱いが非常に難しいと言える.
また,識別器には線形識別器を仮定している研究が多く,計算コストの面からCNNのように膨大なパラメータ数を持つモデルを利用できない場合が多いことも問題として挙げられる.
% また,能動学習の多くでは特徴量が変化しないという状況での問題設定を置いていることが多く,表現学習を同時に行う深層学習を用いた研究は少ない.

\section{本研究の目的}
本研究では,病理画像解析において病理医に対して大きな負担となるアノテーションコストを抑えつつモデルの識別精度を担保するために,
学習の過程で識別器とアノテーターとのインタラクションを通じてラベル付き学習データを増加させていく能動学習を採用したシステムの構築を目的とする.
また、近年の画像認識分野で目覚ましい成果を挙げているConvolutional Neural Network (CNN)を能動学習に適用するための新たな手法として,
Dropoutによってサンプリングされた部分ネットワークの予測の不一致度を利用するQuery-By-Dropout-Predictions(QBDP)を提案し、その有効性を検証する。


\section{本研究の構成}
本論文の構成を以下に示す.\\
第1章で本研究の背景と目的を述べた. \\
第2章では,関連研究と本研究の位置付けについて述べる.\\
第3章では,本研究で提案する手法について述べる.\\
第4,5章では,提案した手法の有効性を確認するために行った実験の目的,設定,結果について述べる.\\
第6章では,本研究の結論,および展望について述べる.

\section{本研究の貢献}
\begin{itemize}
    \item 大規模病理画像解析に能動学習を利用するためのシステムの構築
    \item 深層能動学習において、現実的な計算コストで効率的にクエリを選択するアルゴリズム Query-By-Dropout-Predictionsの提案
    \item 不一致度を定量化するための推論時に、DropoutだけでなくData Augmentationも併用することでさらに効率的なサンプル選択を可能にする手法の提案
    \item 大量の画像パッチからなる病理画像データセットを能動学習で扱うために、ラベルなしデータセットをクエリ選択のたびにサンプリングすることで
    膨大なデータセットに制限を加えることなく現実的な時間でクエリ選択を行う手法の提案
\end{itemize}
