\chapter{Active Learning 能動学習}
能動学習,Active Learning\cite{settles2010active}とは,機械学習における一つの枠組みである.
一般に,教師つき学習において識別精度の高いモデルを学習させるためには膨大なアノテーション付きデータが必要となる.
そこで,能動学習では,得られた大量のラベルなしデータから最もモデル更新に寄与する可能性のあるサンプルを調べ,
それにアノテーションを付与することで,少ないアノテーションコストのもとでいかに高精度な学習モデルを作成するかを目的としている.

能動学習の基本的な流れを以下に示す
\begin{description}
    \item[1.] モデル更新に寄与する可能性のあるサンプルを選択
    \item[2.] 人手,専門家によってアノテーションを付与
    \item[3.] 付与されたアノテーションを利用し教師あり学習
\end{description}

学習が飽和するまでこれらのサイクルを繰り返す

学習の精度はデータ量の対数スケールに比例して上昇
精度向上に要求されるデータ量は指数的に増大

教師あり学習 : 問題集と解答集
半教師付き学習 : 教科書と章末問題
能動学習 : 先生にわからないところを聞く


その選考基準は様々に考案されている.

サンプリングバイアス
    能動学習で得られたデータ集合は実際のデータ集合とは異なる


