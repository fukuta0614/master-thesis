\chapter{実験2 : 病理画像データセットを用いた実験}
\section{概要}
本章では、本研究で提案するシステムを用いて実際に病理画像データセットに対して行った実験について説明する。
MNISTの実験から、Query-by-Committeeを

有効性を検証するためにMNISTに対して行った実験について説明する。
Dropoutによってサンプリングされた各部分ネットワークの出力が未知データに対して分散を持つのかを検証し、
能動学習のクエリ選考基準として有効であるかを確認する。
\section{実験設定}
\subsection{データセットについて}
本実験では、Camelyon Grand Challenge\cite{Camelyon17}にて公開されたCamelyonデータセットに対して実験を行った。
Camelyonデータセットは1000枚のWSIからなる。\todo{図}

\subsection{比較手法について}

\section{実験結果}

\section{考察}
