\chapter{関連研究}

\section{病理画像解析}
本節では、病理画像解析特有の性質と関連研究を述べる。

\subsection{概観}
第1章で述べたように、病理画像とはWSIと呼ばれる巨大なデジタル画像のことを指す。\todo{画像}
病理画像解析のおけるアプリケーションは、大きく分けて3つに分類される\cite{komuraishikawa}。
自動診断による医師の補助、類似画像検索による医師の補助、画像と画像以外の情報との関連性の解析、が挙げられる。
近年の画像認識技術の向上による、自動診断に関する研究が急速に増加している。
本研究でも、自動診断に位置づけられる癌の自動検知タスクに焦点を当てる。

\subsection{パッチベースの分類}
一般に、画像認識タスクにおいて用いられる画像はせいぜい1000×1000以下である。
しかし、数万ピクセル×数万ピクセルに及ぶWSIを一度に学習するのはデータ数、パラメータサイズどちらの面においても現実的ではない。
また、癌などの異常部位というのはWSIにおいてわずか領域にのみ現れることも多いため、基本的にはWSIを大量の画像パッチに分割し、
それぞれを識別した結果を統合することで癌の有無を判定することが多い。
個々の画像パッチを識別するのは、SVM, Convolutional Neural Networkなどが用いられる。

\subsection{特徴抽出}

機械学習において特徴抽出は重要な役割を持つ。
従来はhand-craftedによる特徴量 (SIFT, HLAC, \todo{ちょっと調べる} )が使われていた。
また、テクスチャとしての性質を持つことからテクスチャ解析における手法をしばしば利用することがある。
しかし、近年では一般画像認識の成果を受けてCNNを利用して同時に学習することが多い。
また、pretrained modelによる特量を利用することもある。

\subsection{能動学習の適用}
\com{active + Pathology}
sample, i.i.dじゃないよね.似た領域には似たサンプルがあるよね
enforce diversity 

人手による前処理 + uncertainty samplingのみ\cite{nalisnik2017interactive}
class balancing + query-by-committee\cite{doyle2011active}, 
logistic regression + variance reduction \cite{padmanabhan2014active},
仮説空間の縮小を目的関数にすることで劣モジュラ最適化の枠組みを利用、似たデータが多いことを利用してk-meansでクラスタリング\cite{zhu2014scalable}、


\section{深層能動学習}

この節では、深層能動学習の先行研究について述べる。


