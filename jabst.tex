
病理標本のデジタル画像を撮影し,ディスプレイに表示することで細胞の組織を観察して診断を行うデジタルパソロジー (Digital Pathology)の普及が進んでいる.
病理画像は非常に高解像度で1スライド当たりギガピクセルに及び,組織すべてを隅々まで確認して異常が存在するか診断を行うのは熟練の病理医でも時間を要する.
一方,デジタルパソロジーの普及に伴い組織の画像データが大量に保管されるようになり,画像認識技術の応用によって病理医の診断を補助するための研究が数多く行われている.


近年の画像認識分野において,Convolutional Neural Network (CNN)と呼ばれる多層ニューラルネットワークを用いた機械学習技術による成果が目覚ましい.
従来の画像認識では人手によって特徴量を設計し,その特徴量を用いて識別器を学習していたのに対し,
CNNは学習の過程で訓練データから識別に有効な特徴量を抽出する表現学習と識別器の学習を同時に行うことができる点が特徴的である.

病理画像認識の分野においてもCNNを使用した研究は急速に増加しており,組織の異常や病変の認識,検出の精度は熟練の病理医に匹敵することが示されている.
1スライド当たりギガピクセルに及ぶ病理画像解析では,スライドを複数の画像パッチに分割し,それらに対して識別及びセグメンテーションを行う場合が多い.
我々は,Camelyon17という乳癌のリンパ節転移を検出するタスクにおいて,CNNの共分散特徴量を利用した学習器によって,病理医に近い精度を達成した.
また,機械による識別結果を利用しながら病理医が診断を行った場合,機械のみ,病理医のみによる診断と比較して速度,精度ともに向上したという報告がある.
これらのことから,CNNを病理画像に利用するのは非常に有望であると考えられるが,これらの学習には大量のラベル,またはアノテーションを必要であるという大きな欠点がある.
当然のことながら病理画像の学習データの作成には高度に専門的な知識を必要とするため,ギガピクセルに及ぶ画像の詳細なアノテーションの付与には莫大なコストが伴う.

ラベルが不足した状況に対する機械学習のアプローチとして,半教師付き学習や教師なし学習などが挙げられるが,教師あり学習と比較して精度の点で大きく劣る場合が多い.
そこで本研究では,病理医が支払うコストを最小にしつつ精度を担保するためのアプローチとして,
学習データのラベルを学習の過程で機械とアノテーターとのinteractionを通じて増加させていくActive Learningを病理画像解析に取り入れることを提案する.


