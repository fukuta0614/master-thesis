病理標本を撮像して診断を行うデジタルパソロジー技術の普及に伴い,
画像認識技術によって病理医の診断を補助するための研究が数多く行われている.
一方,高精度な識別率を達成する学習モデルを獲得するには,大規模なラベル付きデータを構築する必要がある.
病理画像等の医療画像の学習データセットの作成には高度に専門的な知識が不可欠であるため,
医師にとって大きな負担となってしまいアノテーションコストが非常に高い.
そこで本研究では,画像群の中からモデルの識別精度向上に寄与する可能性の高いサンプルのみにアノテーションを付与することで
アノテーションコストを抑えつつ高精度なモデルを獲得する能動学習の枠組みを採用する.
また,能動学習を病理画像に適用するための工夫だけでなく,近年の画像認識分野で目覚ましい成果を挙げているConvolutional Neural Network (CNN)を
能動学習に適用するための新たな手法を提案し,大規模病理画像データセットに対して実験を行い有効性を検証した.
