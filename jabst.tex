病理標本を撮像して診断を行うデジタルパソロジーの普及に伴い,
画像認識技術によって病理医の診断を補助する研究が数多く行われている.
一般に,高い識別率を達成する予測モデルを獲得するには,大規模なラベル付きデータセットを構築する必要がある.
しかし,病理画像などの医療画像に対するラベルの付与には高度な専門的知識を要するため,全てのアノテーション作業を医師が行わなければならず,学習データセットの作成コストが非常に高い.

そこで本研究では,アノテーションコストを抑えつつ病理画像に対する高精度な予測モデルを獲得するシステムの構築を目的とする.
そのためのアプローチとして,ラベルが付与されていない画像群からモデルの識別精度向上に寄与する可能性の高い画像を自動的に選択し,人にラベルの付与を依頼する能動学習の枠組みを採用する.
また,近年の画像認識分野で目覚ましい成果を挙げているConvolutional Neural Network (CNN)を能動学習に適用するための新たな手法として,
Dropoutによってサンプリングされた部分ネットワークの予測の不一致度を利用するQuery-By-Dropout-Predictions(QBDP)を提案する.

実験の結果,手書き数字データセットにおいて,QBDPを利用することで既存手法より少ないサンプル数で高精度を達成できることが確認された.
さらに,大規模病理画像データセットに対して本研究で提案するシステムを適用し,有効性を示した.
