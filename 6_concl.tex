\chapter{結論}
本章では,本論文の結論および課題と今後の展望について述べる.

\section{結論}
本研究では,病理画像解析において問題となっている膨大なアノテーションコストを緩和するためのアプローチとして,能動学習を採用した.
また,近年の画像認識で成果を挙げている深層学習を能動学習における識別機として利用するためのクエリ選考基準を考案した.
モデルのバージョン空間を縮小させるサンプルを選択するQuery-By-CommitteeをDropoutによって近似的に再現し,
それらの不一致度を定量化することでCNNのパラメータ更新に寄与するサンプルを効率的に探索が可能であることを,簡易的なデータセットと
大規模病理画像データセットでの実験で検証した.
また,WSIに含まれる大量の画像パッチを能動学習の枠組みで効率的に扱うために,
ラベルなしデータセットを各クエリ問い合わせ毎に一部サンプリングすることで
利用可能なデータを制限しすぎることなく現実的な時間で有効なサンプルを選択する方法を提案した.
さらに,汎化性能を向上させるために通常は訓練時にのみ使用されるData Augmentationを推論時にも使用することで,
特徴抽出のために有効であると考えられるサンプルを取得する手法を提案しその病理画像における顕著な性能を示した.

% 本研究で得られた知見は以下のとおりである.

\section{今後の課題}

\subsection{ラベルなしデータの活用}
本研究では,モデルのバージョン空間を縮小させるラベル付きデータセットを作成するための実験設定になっており,
作成したあとに自由にハイパーパラメータを設定して性能を調整すれば良いのではないかと考えて実験を行った.
しかし,半教師付き学習のようにラベルなしデータも学習に利用することでさらに効率的にサンプル選択をすることができる可能性がある.

\subsection{継続的なfintuneによる学習}
本研究では,モデルが各再学習において過学習してしまうことを恐れ,
ラベル付データが追加された後にモデルをpre-trainedのパラメータに初期化する方針を採用した.
しかし,新たに追加されるラベル付きデータは,その時点でのモデルパラメータにとって最適なものであるため
別のパラメータに初期化してしまった場合は真に最適となっているかは自明ではなくなる.
また,再学習にはその分時間もかかるため,継続的なfinetuneをするための工夫を考える余地は残されている.
