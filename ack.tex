% \markright{謝辞}

% \chapter*{謝辞}
% \addcontentsline{toc}{chapter}{謝辞}
本研究は、東京大学大学院 情報理工学系研究科 創造情報学専攻の原田達也教授のご指導のもとに行われました.
原田達也教授,牛久祥孝講師をはじめとして,原田・牛久研究室のメンバーの皆様から多大な支援を受けて本論文を完成させることができました.

% %原田先生
原田達也教授の研究に対する意識の高さ,視野の広さは研究を行う上で大きな刺激となりました.
非常にお忙しいにも関わらず、的確な指導をしていただき、原田教授のご指摘を通して自分の見逃していた部分に気づかされることも多く、とても勉強になりました。
心より感謝しております.また是非テニスしましょう。

% %牛久先生
牛久祥孝講師は生徒に対する面倒見が良く、研究会での幅広い知識をバックグラウンドにしたアドバイスも的を射ていたのが印象的です。
ベンチャー立ち上げは流れてしまいましたが、もし自分が会社疲れに疲れたらまた相談に乗っていただけたら嬉しいです。

% %中村先生
中村衛助教には、研究室内の事務作業を担当していただき、感謝しております。
何度も勤務表の手続きでご迷惑をおかけして申し訳ありませんでした。

% %学術支援職員
特任研究員のTejero de Pablos Antonioさん、黒瀬優介さんの研究会での鋭い指摘にはいつもハッとさせられました。
技術補佐員の金子葵さん、加治佐美由希さんは我々に快適な研究環境を準備すべく多面にわたり尽力していただき,感謝しております.

% %Doctor

大学院博士課程の森友亮さん、日高雅俊さん、椋田悠介さん、Huang KaiKaiさん、南里卓也さん、兼平篤志さん、葉 浩瑋さん、加藤大晴さん、Tang Yujinさん、
Li Yangさん、金子卓弘さん、Mohammad Reza MOTALLEBIさん、温 穎怡さんには熱心に研究に取り組む姿から研究生活に関する多くのことを学ばせていただきました.
また、研究の進め方の手本を見せていただいたばかりでなく,研究室での充実した生活を提供してくださったことに心より感謝いたします.
椋田さんには機械学習手法に関するお話や、新しい論文に関する議論をしていただき大変勉強になりました。ありがとうございました。
日高さんには自分が修士に入りたての頃にalsarc 2016を通して初歩的なことを色々教えていただきました。また、それ以降もサーバや実装に関する質問を何度もさせていただきました。ありがとうございました。
加藤さんは、GANなどの楽しい論文について共有していただきお話するのが楽しかったです。
また、もう一つの顔であるぱろすけさんは、自分がプログラミングを始めるきっかけになった方でもあるのでそちらに関してもとても感謝しています。

% %Master
修士課程二年目の同期である井関茜さん、齋藤邦章くん、床爪佑司くん、山本将平くん、早川顕生くん、木倉悠一郎くん、高田一真くん、YanTenninくん、石原弘之さん、
渡辺康平くん、張仁彦くん、Hanna Tseranさんは、共に卒業を目指し研究に取り組む同輩として心強く感じさせていただきました.
研究室では楽しい会話を提供してくださり,心の支えになりました。

% %Master 1
修士課程一年目の後輩である岡田英樹くん、高柳臣克くん、唐澤拓己くん、金山哲平くん、佟航くん、町田龍昭くん、荒瀬晃介くん、
新井棟大くん、稲田修也くん、James Borgくん、張徳軒くんは、普段の話相手になっていただけるだけでなく、それぞれが違う興味をもっており良いインスピレーションを受けました。

% %B4
学部4年の石川輝くん、上原康平くん、宇佐美峻くん、河合里咲さん、田中幹大くん、津野蒼くん、野口敦裕くん、松浦寿彦くん
は、自分がB4の頃と比べられないくらい優秀で、全員が真摯に研究に取り組んでおり、自分も負けていられないなといい刺激を受けました。

本論文の完成は,研究室の皆様のご協力がなければ到底なしえないものでした.
最後に,改めて皆様に感謝の意を表すとともに,
どんな時にもあたたかく励まし支え続けてくれた家族に感謝し,
本論文の謝辞とさせていただきます.


